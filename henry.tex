
	\documentclass{article}
\begin{document}

\begin{large}\textbf\center{TRANSLATION TO GENERAL PURPOSE GRAPHICS PROCESSING UNIT.}\end{large}

\section{Introduction.}
{There has been increasing interest in using GPGPU based systems in the school and NVidia/CUDA
seems to have become the default environment for this and this is where most of our experience lies.
A general-purpose GPU (GPGPU) is a graphics processing unit (GPU) that performs non-specialized
calculations that would typically be conducted by the CPU (central processing unit). Ordinarily, the
GPU is dedicated to graphics rendering. GPGPU which is General-purpose computing on GPUs only
became practical and popular after about 2001, with the advent of both programmable shade and
floating point support on graphics processors. Notably, problems involving matrices and/or vectors
especially two, three, or four dimensional vectors were easy to translate to a GPU, which acts with
native speed and support on those types. The scientific computing community’s experiments with
the new hardware began with a matrix multiplication routine (2001). One of the first common
scientific programs to run faster on GPUs than CPUs was an implementation of LU factorization
(2005).
}

\section{Background To The Problem.}
{As both CPU and GPU become employed in a wide range of applications, it has been acknowledged
that both of these processing units (PUs) have their unique features and strengths and hence,
CPU-GPU collaboration is inevitable to achieve high-performance computing. This has motivated
significant amount of research on heterogeneous computing techniques, along with the design of
CPU-GPU fused chips. These early efforts to use GPUs as general-purpose processors required
reformulating computational problems in terms of graphics primitives, as supported by the two major
APIs for graphics processors, OpenGL and DirectX. This cumbersome translation was obviated by
the advent of general-purpose programming languages and APIs such as Sh/Rapid Mind, Brook and
Accelerator. This means that modern GPGPU pipelines can leverage the speed of a GPU without
requiring full and explicit conversion of the data to a graphical form.
}

\section{Problem Statement.}
{There has been a problem of parallel processing multiple evaluation of a sequential task on large
data sets for example when you have code involving lots of large (especially nested) loops, it became
a must to calls for parallelization of that peace of code anyway. If it is a code that executes within
the loops then it becomes a must for it to be reduced to a fairly simple algorithm.Putting that a
side, there is also a problem were you have a large dataset that fits, or needs to be made to fit into
the memory structures of the hardware you have. If your code involves manipulating large matrices,
all these above problems makes a call and need of GPGPU for them to be solved.
}

\section{Objectives.}
{To develop a high-performance computing system that can do automatic translation to general
purpose computing on an FFT mapping on GPU for radar processing application.

To collect and analyze the requirements about the relevance and feasibility of the Automatic translation
to GPGPU on an FFT mapping on GPU for radar processing application.

To implement the FFT mapping on GPU for radar processing application.

To test and validate the application.
}

\section{Research Scope.}
{This research is based on Efficient FFT mapping on GPU for radar processing application. General purpose
multiprocessors (as, in our case, Intel Ivy Bridge and Intel Haswell) increasingly add GPU
computing power to the former multicore architectures. When used for embedded applications (for
us, Synthetic aperture radar) with intensive signal processing requirements, they must constantly
compute convolution algori thms, such as the famous Fast Fourier Transform. Due to its ”fractal”
nature (the typical butterfly shape, with larger FFTs defined as combination of smaller ones with
auxiliary data array transpose functions), one can hope to compute analytically the size of the largest
FFT that can be performed locally on an elementary GPU compute block. Then, the full application
must be organized around this given building block size. Now, due to phenomena involved in the
data transfers between various memory levels across CPUs and GPUs, the optimality of such a
scheme is only loosely predictable (as communications tend to overcome in time the complexity of
computations). Therefore  mix of (theoretical) analytic approach and (practical) runtime validation
is here needed. This occurs at both stage, first at the level of deciding on a given elementary FFT
block size, then at the full application level.
}

\section{References.}
{Proceedings of the international Symposium on Wearable Computing 2002(ISWC2002),Washington, USA, 7-10 October 2002, pp.83-89.

Mark Harris. Mapping computation concepts to GPUs. In ACM SIGGRAPH 2005 Courses(Los Angeles,Califonia,31 July-4 August 2005). J.Fujji, Ed.SIGGRAPH’’05.ACM Press, New York, NY, 50.

Dominik Goddeke, Robert Strzodka, and Stefan Turek. Accelerating Double Precision(FEM) Simulations with (GPUS). Proceedings of ASIM 2005-18th Symposium on Simulation Technique, 2005.
}

\end{document}
